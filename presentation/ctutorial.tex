\documentclass{beamer}
\usepackage{listings}
\usetheme{default}

\lstset{numbers=left,
    numberstyle=\tiny,
    stepnumber=1,
    numbersep=5px,
    keywordstyle=\color{red},
    stringstyle=\color{green},
    commentstyle=\color{blue}}

\title{Interfacing C with OCaml}
\author{Niki Yoshiuchi}
\date{\today}

\begin{document}

\begin{frame}[plain]
    \titlepage
\end{frame}

\begin{frame}{Overview}
\begin{tabular}{p{0.4\textwidth}p{0.5\textwidth}}
\begin{itemize}
    \item Why?
    \item A Quick Example
    \begin{itemize}
        \item C Code
        \item OCaml Code
        \item Compiling
    \end{itemize}
    \item Function signatures
    \begin{itemize}
        \item Arity $>$ 5
    \end{itemize}
    \item value type
    \begin{itemize}
        \item Integral types
        \item Blocks
        \item Structured data
    \end{itemize}
\end{itemize} &

\begin{itemize}
    \item Linking
    \begin{itemize}
        \item Static
        \item Dynamic
    \end{itemize}
    \item Advanced features
    \begin{itemize}
        \item Exceptions
        \item Callbacks
    \end{itemize}
\end{itemize} \\
\end{tabular}
\end{frame}

\begin{frame}{Why?}
\begin{itemize}
    \item Bindings to C libraries
    \item Multi-core threading
    \begin{itemize}
        \item If the C code doesn't require the OCaml runtime we can release the lock
    \end{itemize}
    \item Breaking the type system
    \begin{itemize}
        \item If you are going to do this, it's probably better to use the Obj module
    \end{itemize}
\end{itemize}
\end{frame}

\begin{frame}[fragile]{A Quick Example: C Code}
\begin{lstinputlisting}[language=C]{../quick/stubs.c}
\end{lstinputlisting}
\end{frame}

\begin{frame}[fragile]{A Quick Example: OCaml Code}
\begin{lstinputlisting}[language=ML]{../quick/main.ml}
\end{lstinputlisting}
\end{frame}

\begin{frame}[fragile]{A Quick Example: Compiling}
\begin{lstlisting}
gcc -I /usr/local/lib/ocaml -c stubs.c
ocamlc -c main.ml
ar rcs libstubs.a stubs.o
ocamlc -o main.byte -custom libstubs.a main.cmo
\end{lstlisting}
\end{frame}

\begin{frame}[fragile]{Function Signatures}
The OCaml code is simple - declare your function using the \emph{external} keyword,
 specify its type and set it equal to the name of the C implementation.
\begin{itemize}
    \item OCaml:
    \begin{lstlisting}[language=ML]
    external name : type = "C-function-name"
    \end{lstlisting}
    \item C:
    \begin{lstlisting}[language=C]
    CAMLprim value name(value arg1, value arg2)
    \end{lstlisting}
\end{itemize}
\end{frame}

\begin{frame}[fragile,shrink=15]{Arity $>$ 5}
For functions with Arity n $>$ 5, we must implement two C functions - one
with n arguments and one with an array.  The former is used for native compilation,
the latter for byte code compilation.
\begin{itemize}
    \item OCaml:
    \begin{lstlisting}[language=ML]
    external name : type = "byte-code-name"
        "native-code-name"
    \end{lstlisting}
    \item C (native):
    \begin{lstlisting}[language=C]
    CAMLprim value native_code(value arg1, value arg2,
    value arg3, value arg4, value arg5, value arg6)
    \end{lstlisting}
    \item C (byte):
    \begin{lstlisting}[language=C]
    CAMLprim value byte_code(value *argv, int argc) {
        native_code(argv[0], argv[1], argv[2],
        argv[3], argv[4], argv[5]);
    }
    \end{lstlisting}
\end{itemize}
\end{frame}

\begin{frame}{value type}
All variables passed between OCaml and C are of type \emph{value}.  A \emph{value} can be:
\begin{itemize} \item A integral type (unboxed) \item A pointer to a heap allocated object
    \item An pointer to a object allocated outside of the heap
\end{itemize}
There are two macros to help you determine what type is encapsulated in a \emph{value}
\begin{itemize}
    \item Is\_long(v) - returns true if v is an integral type
    \item Is\_block(v) - returns true if v is a pointer to a block
\end{itemize}
\end{frame}

\begin{frame}{integral types}
There are a number of macros for retreiving and storing integral types.
They are of the form \emph{Val\_type} (think "value of type") and \emph{Type\_val} (think "type of value")
\begin{tabular}{p{0.4\textwidth}p{0.5\textwidth}}
    \begin{itemize}
        \item Val\_long(l)
        \item Val\_int(i)
        \item Val\_bool(b)
        \item Val\_true
        \item  Val\_false
        \item Val\_unit
    \end{itemize} &
    \begin{itemize}
        \item Long\_val(v)
        \item Int\_val(i)
        \item Bool\_val(v)
    \end{itemize} \\
\end{tabular}
\end{frame}

\begin{frame}[fragile]{Going back to our quick example}
Let's take another look at our quick example to see how these macros are used:
\begin{lstinputlisting}[language=C]{../quick/stubs.c}
\end{lstinputlisting}
\end{frame}

\begin{frame}{Blocks}
If a \emph{value} is a block, then it has a tag (accessible via the \emph{Tag\_val(v)} macro) which will
return one of the following values:
\begin{tabular}{| l | l |} \hline
    Tag & Description \\ \hline\hline
    0 to No\_scan\_tag-1 & A structured block.  Each field is a \emph{value}. \\ \hline
    Closure\_tag & A closure representing a functional value. \\ \hline
    String\_tag & A character string. \\ \hline
    Double\_tag & A double precision float. \\ \hline
    Abstract\_tag & An abstract datatype. \\ \hline
    Custom\_tag & A custom datatype. \\ \hline
\end{tabular}
\end{frame}

\begin{frame}{Doubles, strings, Ints}
Doubles, strings, Int32s, Int64s and Nativeints are all stored as pointers to heap allocated objects.
They are accessible via a set of macros of the form \emph{Type\_val}:
\begin{itemize}
    \item Double\_val(v)
    \item String\_val(v)
    \item Int32\_val(v)
    \item Int64\_val(v)
    \item Nativeint\_val(v)
\end{itemize}
\end{frame}

\begin{frame}{Allocating doubles, strings and Ints}
In order to pass a double, string or Int32/Int64/Nativeint from C to OCaml, you must allocate an
object on OCaml's heap.
\begin{itemize}
    \item caml\_alloc(n, t) - Allocates a block with tag \emph{t} and size \emph{n}
    \item caml\_alloc\_string(n) - Allocates a string of length n
    \item caml\_copy\_string(s) - Copies the null-terminated string s
    \item caml\_copy\_double(d)
    \item caml\_copy\_int32(i), caml\_copy\_int64(i), caml\_copy\_nativeint(i)
\end{itemize}
\end{frame}

\begin{frame}{Garbage Collection}
If your C code manipulates any heap allocated objects or performs any allocations, you must
register it with the garbage collector.  OCaml provides a number of macros:
\begin{itemize}
    \item CAMLparam0 to CAMLparam5 - registers 0 to 5 parameters
    \item CAMLxparam1 to CAMLxparam5 - for registering additional parameters for functions with arity $>$ 5
    \item CAMLlocal1 to CAMLlocal5 - for declaring local \emph{value} variables
    \item CAMLlocalN(x, n) - for declaring an array of \emph{values} of size \emph{n}
    \item CAMLreturn0, CAMLreturn(x) and CAMLreturnT(t, x) - these replace the \emph{return} statement
\end{itemize}
\end{frame}

\begin{frame}{Garbage Collection}
\begin{itemize}
    \item CAMLparam\# and CAMLxparam\# should be the first things called in your function.  All your function's
    parameters should be registered using them.  If you have 3 parameters, then use CAMLparam3, etc.
    \item CAMLlocal\# and CAMLlocalN should be used to declare any local \emph{value} variables used in yoru function.
    \item CAMLreturn should replace all the \emph{return} statements in your function.  This is required if you used
    any of the CAMLparam or CAMLlocal functions.
\end{itemize}
\end{frame}

\begin{frame}[fragile]{Double example}
\begin{lstinputlisting}[language=C]{../double/stubs.c}
\end{lstinputlisting}
\end{frame}

\begin{frame}{Structured Data}
\begin{itemize}
    \item Variants
    \item Records and tuples
    \item Lists
    \item Arrays
\end{itemize}
\end{frame}

\begin{frame}{Variants}
\begin{itemize}
    \item Constant constructors are represented as unboxed integers.  The value is the order in which the constructor
    appears, starting from 0.
    \item Non-constant constructors are represented as blocks.  The tag for each block is the order in which the
    non-constant constructor appears, starting from 0.
\end{itemize}
\end{frame}

\begin{frame}[fragile,shrink=10]{Variant Example}
\begin{lstinputlisting}[language=ML]{../variant/main.ml}
\end{lstinputlisting}
\end{frame}

\begin{frame}[fragile,shrink=20]{Variant example}
\begin{lstinputlisting}[language=ML]{../variant/stubs.c}
\end{lstinputlisting}
\end{frame}

\begin{frame}{Tuples and Records}
Tuples and records are both represented by 0 tagged blocks.  Fields are accessed in numerical order using the
\emph{Field(v, n)} macro.  \emph{Field(v, n)} returns an lvalue and can be used to both set and get the field.
Tuples and records can be allocated using the \emph{caml\_alloc\_tuple(n)} function.
\end{frame}

\begin{frame}[fragile]{Record example}
\begin{lstinputlisting}[language=ML]{../record/main.ml}
\end{lstinputlisting}
\end{frame}

\begin{frame}[fragile]{Record example}
\begin{lstinputlisting}[language=C]{../record/stubs.c}
\end{lstinputlisting}
\end{frame}

\begin{frame}{Lists}
\begin{itemize}
    \item $[]$ is a Val\_int of 0
    \item Blocks with tag 0
    \item Block size of 2
    \item Field(v, 0) contains the head
    \item Field(v, 1) contains the tail
\end{itemize}
\end{frame}

\begin{frame}[fragile]{List example}
\begin{lstinputlisting}[language=ML]{../list/main.ml}
\end{lstinputlisting}
\end{frame}

\begin{frame}[fragile,shrink=5]{List example}
\begin{lstinputlisting}[language=C]{../list/stubs.c}
\end{lstinputlisting}
\end{frame}

\begin{frame}{Arrays}
\begin{itemize}
    \item Block with tag 0
    \item Use the \emph{Wosize\_val(v)} macro to get the size of the array
    \item Use Field(v, n) to get or set the nth element
    \item caml\_alloc\_tuple(n) to allocate an array of size n
    \item Double arrays
    \begin{itemize}
        \item Block with tag Double\_array\_tag
        \item Use Double\_field(v, n) to get the nth element
        \item Store\_double\_field(v, n, d) to set the nth element
        \item caml\_alloc(n, Double\_array\_tag) to allocate an array of size n
    \end{itemize}
\end{itemize}
\end{frame}

\begin{frame}[fragile]{Array example}
\begin{lstinputlisting}[language=ML]{../array/main.ml}
\end{lstinputlisting}
\end{frame}

\begin{frame}[fragile,shrink=5]{Array example}
\begin{lstinputlisting}[language=C]{../array/stubs.c}
\end{lstinputlisting}
\end{frame}

\begin{frame}[fragile]{Double array example}
\begin{lstinputlisting}[language=ML]{../darray/main.ml}
\end{lstinputlisting}
\end{frame}

\begin{frame}[fragile,shrink=5]{Double array example}
\begin{lstinputlisting}[language=C]{../darray/stubs.c}
\end{lstinputlisting}
\end{frame}

\end{document}
